\documentclass[17pt,aspectratio=1610,cjk, xelatex]{beamer}


\usepackage{fontspec}

\usepackage{zxjatype}
\usepackage{txfonts}
\usepackage{amsmath,amssymb}
\usepackage{listings}

%% change fonts you like :)
\setjamonofont{ipag.ttf}
\setjamainfont{migmix-1p-regular.ttf}
\setsansfont[Scale=MatchLowercase]{Tahoma}
\setmonofont[Scale=MatchLowercase]{Tahoma}

\usefonttheme{professionalfonts}
\AtBeginShipoutFirst{\special{pdf:tounicode EUC-UCS2}}
\setbeamertemplate{navigation symbols}{}

\usetheme{GE}

\title{すたいるのてすと}
\author{s10018}
\date{きょう}

\begin{document}

\maketitle{}

\begin{frame}
  \frametitle{てすと}
  
  \begin{itemize}
      \item かじょうがき1
      \item かじょうがき2
  \end{itemize}

  \begin{description}
      \item[日本] かじょうがき1
      \item[English] かじょうがき2
  \end{description}
  
  \begin{block}{ぶろっく}
    これはぶろっくです

    ぶろっくなんです?
  \end{block}
\end{frame}

\begin{frame}[fragile]
  \frametitle{てすと2}

  \begin{exampleblock}{せつめいぶろっく}
    これはせつめいぶろっく

    せつめいにつかえる?

\begin{lstlisting}[language=Python, basicstyle=\small\ttfamily]
  for i in range(10):
      print i
\end{lstlisting}
  \end{exampleblock}

  \begin{alertblock}{アラートブロック}
    注意!!
  \end{alertblock}

\end{frame}


\begin{frame}{証明しない}
\begin{theorem}
  $A + 0 = A$
\end{theorem}
\begin{proof}
  定義から$A + 0 = A$はきっと成り立ちます
\end{proof}
\end{frame}


\end{document}

%%% Local Variables: 
%%% mode: japanese-latex
%%% TeX-master: "."
%%% End: 
