\documentclass[17pt,aspectratio=169,dvipdfmx]{beamer}

\usepackage[T1]{fontenc}
\usepackage{libertine}
\renewcommand*\familydefault{\sfdefault}
\usepackage{GE} 

\title{Sample Slide}
\author{Name}
\date{Date}

\begin{document}

\maketitle{}

\begin{frame}{frame1}
  \begin{itemize}
      \item A
      \item B
    \begin{enumerate}
        \item AA
        \item BB
    \end{enumerate}
      \item C
  \end{itemize}
\end{frame}

\begin{frame}{frame2}

  {\color{base}{base color}}

  {\color{fontc}{font color}}

  {\color{sub1}{sub color 1}}

  {\color{sub2}{sub color 2}}

  \begin{table}[b]
    \centering
    \begin{tabular}{ccc}
      \hline
      A & B & C \\\hline
      1 & 2 & 3 \\\hline
    \end{tabular}
    \caption{Test Table}
  \end{table}

\end{frame}

\begin{frame}{frame3}

  \begin{theorem}<1->
     There exists an infinite set.
   \end{theorem}

   \begin{proof}<2->
     This follows from the axiom of infinity.
   \end{proof}

   \begin{example}<3->[Natural Numbers]
     The set of natural numbers is infinite.
   \end{example}

\end{frame}

\begin{frame}{frame4}

  \begin{block}{Definition}
    A \alert{set} consists of elements.
  \end{block}

  \begin{alertblock}{Wrong Theorem}
    $1=2$.
  \end{alertblock}

\end{frame}

\end{document}

%%% Local Variables: 
%%% mode: japanese-latex
%%% TeX-master: t
%%% End: 
